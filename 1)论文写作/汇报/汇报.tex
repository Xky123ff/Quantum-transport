\documentclass{beamer}

% --- 包设置 ---
\usepackage{ctex} % 提供中文支持

% --- 主题与颜色 ---
% 你可以尝试其他主题,例如:Boadilla, Madrid, Warsaw, Singapore, etc.
\usetheme{Madrid}
\usecolortheme{default}

% --- 文稿信息 ---
\title{节流过程的原理及其关键应用}
\author{陈嗣杰}
\institute{课程:热力学与统计物理}
\date{\today}

\begin{document}

% --- 标题页 ---
\begin{frame}
  \titlepage
\end{frame}

% --- 目录页 ---
\begin{frame}
  \frametitle{目录}
  \tableofcontents
\end{frame}


% --- 第一部分:引言与基本原理 ---
\section{引言与基本原理}

\begin{frame}
  \frametitle{什么是节流过程?}
  \begin{itemize}
    \item<1-> \textbf{定义:} 流体在\alert{绝热}条件下,流经一个阀门、多孔塞或其他阻碍物,导致其\alert{压强显著降低}的过程。
    \vfill
    \item<2-> \textbf{核心特征(理想模型):}
    \begin{itemize}
      \item 绝热过程:与外界没有热量交换 ($Q=0$)。
      \item 对外不做功:系统边界没有移动 ($W=0$)。
      \item<3-> \textbf{\alert{最重要的结论:焓守恒 ($H_1 = H_2$)}}。这是一个等焓过程。
    \end{itemize}
  \end{itemize}
\end{frame}

\begin{frame}
  \frametitle{焦耳-汤姆孙效应 (Joule-Thomson Effect)}
  \begin{itemize}
    \item \textbf{现象:} 节流过程中,气体的\alert{温度会发生变化}(可能升高,也可能降低)。
    \vfill
    \item \textbf{焦耳-汤姆孙系数 ($\mu_{JT}$):}
    \begin{itemize}
      \item 定义式:$\mu_{JT} = \left(\frac{\partial T}{\partial P}\right)_H$
      \item $\mu_{JT} > 0$:\alert{节流致冷}。压强降低,温度也随之降低。这是绝大多数应用的基础。
      \item $\mu_{JT} < 0$:节流致热。压强降低,温度反而升高。
      \item $\mu_{JT} = 0$:温度不变,此时的温度点称为“转化温度”。
    \end{itemize}
    \vfill
    \item \textbf{转化温度曲线:}
    \begin{itemize}
      \item 实际气体在T-P图上存在一个转化曲线。
      \item 只有在\alert{转化温度以下的区域},节流才能致冷。
    \end{itemize}
  \end{itemize}
\end{frame}


% --- 第二部分:核心应用领域 ---
\section{核心应用领域}

\begin{frame}
  \frametitle{应用一:制冷与空调技术}
  \begin{columns}[T]
    % 左侧栏:文字说明
    \begin{column}{0.5\textwidth}
      \begin{itemize}
        \item \textbf{核心地位:} 节流是蒸气压缩制冷循环的\alert{四大关键环节}之一。
        \vfill
        \item \textbf{应用实例:}
        \begin{itemize}
            \item 家用冰箱、空调
            \item 汽车空调
            \item 大型冷库等
        \end{itemize}
      \end{itemize}
    \end{column}
    % 右侧栏:循环过程
    \begin{column}{0.5\textwidth}
      \textbf{循环过程简述:}
      \begin{enumerate}
        \item<1-> \textbf{压缩:} 压缩机做功,低温低压 $\rightarrow$ 高温高压气体。
        \item<2-> \textbf{冷凝:} 在冷凝器中放热,高温高压气体 $\rightarrow$ 高压液体。
        \item<3-> \textbf{\alert{节流}:} 流经\alert{膨胀阀},压强和温度骤降,高压液体 $\rightarrow$ 低温低压液汽混合物。
        \item<4-> \textbf{蒸发:} 在蒸发器中吸热,低温低压混合物 $\rightarrow$ 低温低压气体,实现制冷。
      \end{enumerate}
    \end{column}
  \end{columns}
\end{frame}

\begin{frame}
  \frametitle{应用二:气体的液化}
    \begin{itemize}
    \item \textbf{基本原理:} 利用\alert{深度节流致冷},使气体温度降低到其沸点以下,从而液化。
    \vfill
    \item \textbf{林德液化循环 (Linde-Hampson Cycle):}
    \begin{itemize}
      \item \textbf{关键技术:} 结合了 \textbf{\alert{节流致冷}} 与 \textbf{\alert{回流换热}}。
      \item \textbf{流程:} 气体被压缩预冷 $\rightarrow$ 通过节流阀膨胀降温 $\rightarrow$ 冷却后的气体通过换热器去冷却后续的高压气体 $\rightarrow$ \alert{反复循环,逐级降温} $\rightarrow$ 最终液化。
    \end{itemize}
    \vfill
    \item \textbf{应用实例:}
    \begin{itemize}
        \item \textbf{液氮 ($LN_2$):} 科研、医疗、生物样本保存。
        \item \textbf{液氧 ($LO_X$):} 航天、医疗、工业。
        \item \textbf{液化天然气 (LNG):} 能源储存与运输。
    \end{itemize}
  \end{itemize}
\end{frame}


% --- 第三部分:其他重要应用 ---
\section{其他重要应用}

\begin{frame}
  \frametitle{低温技术与工业控制}
  \begin{block}{低温技术 (Cryogenics)}
    \begin{itemize}
        \item \textbf{焦耳-汤姆孙制冷机:} 以节流过程为核心,可达到极低温度(如液氦温区)。
        \item \textbf{应用:} 冷却超导磁体(MRI、粒子加速器)、红外探测器等。
    \end{itemize}
  \end{block}
  \vfill
  \begin{block}{工业过程控制}
    \begin{itemize}
        \item \textbf{蒸汽动力系统:} 使用节流阀调节蒸汽流量和压力,控制发电机功率。
        \item \textbf{天然气处理:} 利用节流降温分离天然气中的重烃组分。
    \end{itemize}
  \end{block}
\end{frame}


% --- 第四部分:总结 ---
\section{总结}

\begin{frame}
  \frametitle{总结}
  \begin{itemize}
    \item \textbf{回顾核心:} 节流过程是一个\alert{等焓过程},其应用的核心是\alert{焦耳-汤姆孙效应}(温度变化)。
    \vfill
    \item \textbf{概括应用:} 从维持我们日常生活舒适的\alert{冰箱空调},到支撑尖端科研的\alert{气体液化}和\alert{低温技术},节流过程是现代工业和科技中不可或缺的一环。
    \vfill
    \item \textbf{结束语:}
    \begin{center}
        一个看似简单的物理过程,\\
        却构成了现代文明高效运转的关键技术基石。
    \end{center}
  \end{itemize}
\end{frame}


% --- 结束页 ---
\begin{frame}
    \begin{center}
        \Huge\bfseries
        致谢 \& 提问环节
    \end{center}
\end{frame}


\end{document}