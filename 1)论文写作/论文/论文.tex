\documentclass[aps,prl,reprint,groupedaddress]{revtex4-2}

% --- Packages ---
\usepackage{graphicx} % Needed for figures
\usepackage{dcolumn}  % Align table columns on decimal point
\usepackage{bm}       % bold math
\usepackage{amsmath}  % For math environments
\usepackage{hyperref} % For hyperlinks in the PDF

\begin{document}

\preprint{APS/123-QED}

\title{On the Factors Influencing the Quantization Accuracy in the Quantum Anomalous Hall Effect}

\author{Xiyang Xu}
% \email{Your-Email@university.edu}
\affiliation{
 Department of Physics, [China Three Gorges University], [Yichang, 443000], [China]
}

\date{2025 September 22}

\begin{abstract}
For the quantum anomalous Hall effect, why is the experimentally observed quantized Hall resistance accompanied by a non-zero longitudinal resistance?Currently,multiple mechanisms have been proposed to explain this phenomenon,including thermal activation of bulk carriers,variable range hopping in the presence of disorder,and edge state backscattering due to magnetic impurities.However,a comprehensive understanding remains elusive.Additionally,the role of sample geometry and contact configuration in influencing the measured resistances is not fully understood.Further experimental and theoretical investigations are necessary to clarify these issues and achieve a complete understanding of transport phenomena in quantum anomalous Hall systems.
\end{abstract}

\maketitle

\section{I. Different mechanisms }


In an ideal quantum anomalous Hall (QAHE) insulator, the bulk is fully insulating, and charge transport is exclusively carried by one-dimensional (1D) chiral edge states. These states are topologically protected from backscattering, which should result in a strictly zero longitudinal resistance ($R_{xx}=0$) and a Hall resistance ($R_{xy}$) precisely quantized to $h/e^2$. However, this ideal scenario is disrupted by material imperfections in realistic systems, such as magnetically doped topological insulators. The physical origins of the experimentally observed non-zero $R_{xx}$ are a central topic of current research. The main theoretical explanations can be categorized as follows.

\subsection{A. Dissipation from Bulk Inhomogeneity}

This is one of the most widely accepted models. It assumes that realistic QAHE materials are far from homogeneous.

Charge Puddles due to factors like the substrate, material defects, or the random distribution of magnetic dopants (e.g., Cr or V), the electrostatic potential within the material fluctuates, leading to the formation of localized electron and hole "puddles." These puddles are local conductive regions. Although surrounded by the insulating QAHE "sea," they can serve as intermediate "stepping stones" for carriers to traverse the bulk from one edge to another.

Physical Mechanism Electrons can leverage these bulk localized states to tunnel between opposite edges via a process known as \textbf{variable-range hopping (VRH)}. This process effectively breaks the perfect 1D conduction of the edge channels, constitutes a form of backscattering, and consequently generates a non-zero longitudinal resistance $R_{xx}$ while causing deviations from perfect Hall quantization. Temperature-dependent transport measurements, which often reveal characteristic hopping behavior for $R_{xx}$, provide strong support for this model.

\subsection{B. Edge-State Related Scattering Mechanisms}

This category of theories focuses on imperfections associated with the edge states themselves, even assuming an ideal insulating bulk.

Direct Inter-Edge Tunneling if the sample is sufficiently narrow, or if a geometric constriction brings two counter-propagating edge states into close proximity, direct quantum tunneling between them can occur. This provides a clear channel for backscattering.

Edge Disorder and Reconstruction The physical edges of a sample are not atomically smooth. Disorder or atomic reconstruction at the edges can create localized charge traps or puddles nearby. Propagating electrons in the edge channels can be captured and subsequently released by these localized states, a process that can effectively scatter them into the counter-propagating channel, thus contributing to dissipation.

\subsection{C. Percolation Model}

This model can be viewed as a more sophisticated extension of the bulk inhomogeneity theory, describing the phenomenon in the context of a topological phase transition.

Topological Domains Inhomogeneities in magnetic doping can cause regions with different Chern numbers (e.g., $C=+1$, $C=-1$, and $C=0$) to coexist within the sample. These regions are known as topological domains, and conducting chiral states exist at the domain walls separating them.

Physical Mechanism As the Fermi level is tuned, for instance by a gate voltage, the size and connectivity of these domains change. Near the topological phase transition, a continuous path of trivial ($C=0$) or oppositely chiral ($C=-1$) domains may form, spanning the sample and connecting the two edges. This "percolation path" effectively shorts the bulk, leading to a peak in $R_{xx}$. Even deep within the QAHE plateau regime, isolated non-quantized puddles may persist. While they do not form a percolation network, they still contribute to a finite residual $R_{xx}$ through hopping transport, consistent with the bulk inhomogeneity model.




\section{II. Core Application Areas}

\section{III. Other Significant Applications}

\section{IV. Conclusion and Outlook}


\end{document}